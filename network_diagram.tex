% Home Network Diagram
% Author: Michael Bugert
% License: CC BY 4.0
\documentclass{standalone}

\usepackage[utf8]{inputenc}
\usepackage[sfdefault]{noto}
\usepackage[T1]{fontenc}
\usepackage{enumitem}
\usepackage[fixed]{fontawesome5}
\usepackage{etoolbox}
\usepackage{tcolorbox}
\usepackage{tikz}
\usepackage[hidelinks]{hyperref}

\tcbuselibrary{skins,raster}
\usetikzlibrary{positioning,matrix,fit,decorations.pathreplacing}

\graphicspath{{img/}}

% colors for subnetworks
\definecolor{subnet-none}{HTML}{000000}
\definecolor{subnet-home}{HTML}{E57400}
\definecolor{subnet-guest}{HTML}{14C300}
\definecolor{subnet-vpn}{HTML}{AD0022}
\definecolor{docker-bright}{HTML}{0A97C4}
\definecolor{docker-dark}{HTML}{088AB4}

\tcbset{
	skin=enhanced,
	boxrule=0.1em,
	left=0.25em,
	right=0.25em,
	% box with hostname and device specs
	host description box/.style={
		width=19em,
		sidebyside,
		righthand ratio=0.4,
		fonttitle=\bfseries\large,
		fontupper=\small,
		fontlower=\small
	},
	% tcbraster for network interfaces
	interfaces raster/.style={
		raster columns=1,
		raster width=6.75em,
		% fix raster height to accomodate for fontawesome icon height differences
		raster height=4.25em,
		raster row skip=0.4em,
		raster every box/.style={
			fontupper=\ttfamily,
			left=0em,
			right=0em,
			top=.15em,
			bottom=.15em,
			valign=center
		}
	},
	% tcbraster for host services
	services raster/.style={
		raster columns=1,
		boxrule=0.1em,
		raster width=14em,
		raster row skip=0.4em,
		raster every box/.style={
			left=0em,
			right=0em,
			top=.15em,
			bottom=.15em,
			fontupper=\small,
			boxrule=0.1em
		}
	},
	% box around host services:
	% Never drawn, only used as a container. Width is set to the same width as "services". If width is not specified, the box would extend much further towards the right because the default width is \linewidth (uncomment 'blankest' to see it).
	services box/.style={
		blankest,
		width=14em
	},
	% box around dockerized host services
	dockerized box/.style={
		%title={Dockerized\hfill\includegraphics[height=12pt]{services/docker}},
		title={Dockerized},
		fonttitle=\bfseries,
		colframe=docker-dark,
		colback=docker-bright,
		left=0.25em,
		right=0.25em
	},
	% box around devices with less details
	devices box/.style={
		width=20em,
		fonttitle=\bfseries\large,
		fontupper=\small,
		attach boxed title to top left,
		boxed title style={frame hidden, opacityback=0.0, left=0em},
		colback=white,
		coltitle=black!75!white
	},
	% tcbraster for devices with less details
	devices raster/.style={
		raster columns=1,
		boxrule=0.1em,
		raster width=20em,
		raster row skip=0.4em,
		raster every box/.style={
			left=0em,
			right=0em,
			top=.15em,
			bottom=.15em,
			fontupper=\small,
			boxrule=0.1em,
			height=3em,
			valign=center
		}
	}
}

\tikzset{
	single icon/.style={
		font={\fontsize{34}{38}\selectfont}
	},
	host description/.style={
	},
	% box drawn around all bits defining a host using tikz fit
	host box/.style={
		inner sep=0.5em,
		draw=black!75!white,
		fill=white,
		rounded corners,
		line width=0.1em
	},
	% box drawn around all bits defining a region using tikz fit
	region box/.style={
		inner sep=2em,
		rounded corners=2em,
		line width=0.1em,
		fill=#1!5!white
	},
	region descriptor/.style={
		outer sep=0,
		execute at begin node={\fontsize{7em}{7em}\selectfont\scshape},
		color=#1!15!white
	},
	% tag for marking devices or services as DNS, DHCP, etc. with a customized color
	tag/.style={
		draw=none,
		fill=#1,
		rectangle,
		rounded corners=2pt,
		outer sep=3pt,
		font=\bfseries\scriptsize,
		text=white
	},
	% connection types
	generic/.style={
		solid,
		line width=0.2em,
		rounded corners
	},
	ethernet/.style={
		generic
	},
	wifi/.style={
		generic,
		dashed,
	},
	vpn/.style={
		generic,
		decoration={
			waves,
			radius=0.8em,
			segment length=.55em,
			angle=30
		}
	}
}

% commands to show device and service icons at fixed size, etc.
\newcommand{\interfaceImage}[1]{%
	\raisebox{-.1\height}{\includegraphics[height=12pt]{#1}}%
}
\newcommand{\deviceImage}[1]{%
	\includegraphics[width=5em, height=3em, keepaspectratio]{devices/#1}%
}
\newcommand{\deviceImageInline}[1]{%
	\raisebox{-.25\height}{%
		\parbox[t]{2em}{%
			\centering%
			\includegraphics[width=2em, height=2em, keepaspectratio]{devices/#1}%
		}%
	}%
}
\newcommand{\ip}[2]{%
	\textbf{\textcolor{#2}{#1}}
}

% clickable URLs shown without protocol
\newcommand{\https}[1]{%
	\href{https://#1}{\nolinkurl{#1}}%
}
\newcommand{\http}[1]{%
	\href{http://#1}{\nolinkurl{#1}}%
}

% usage: \service{image}{textual description}{url}
\newcommand{\service}[3]{%
	\ifblank{#1}{%
		\hspace{2em}%
	}{%
		\raisebox{-.3\height}{%
			\includegraphics[height=2em]{services/#1}%
		}\hspace{.5em}%
	}%
	\parbox[m]{10em}{%
		#2%
		\ifblank{#3}{}{%
			\newline%
			\tiny\textcolor{gray}{#3}%
		}%
	}
}

\newcommand{\planned}{\hfill\scriptsize\textcolor{black!60}{\textit{(planned)}}}
\newcommand{\temporary}{\hfill\scriptsize\textcolor{black!60}{\textit{(temporary)}}}

\begin{document}
	
% remember picture is needed to retain positions of tcolorboxes (otherwise all edges will be misplaced)
\begin{tikzpicture}[remember picture]
	\pgfdeclarelayer{regions}
	\pgfdeclarelayer{hostboxes}
	\pgfsetlayers{regions,hostboxes,main}
	
	% --------------- start internet ---------------
	\node[single icon, label=above:Internet] (internet) {\faIcon{globe}};
	\path (internet) -- (0,-2) coordinate[name=below-internet];
	% --------------- end internet ---------------	
	
	% --------------- start ISP ---------------
	\node[single icon, left=16em of below-internet, label=below:ISP] (isp-home) {\faIcon{building}};
	% --------------- end ISP ---------------
	
	% --------------- start router ---------------
	% host description
	\node[host description, below left=12em of isp-home] (router-description) {%
		\begin{tcolorbox}[
			adjusted title=router,
			host description box
		]			
			\ip{192.168.0.1}{subnet-home}
		
			\tcblower
			%\deviceImage{my-router-image}
			
			HomeBox CM 3000
		\end{tcolorbox}
	};
	
	% services
	\node[
		below=0em of router-description.south west,
		anchor=north west
	] (router-services) {
	};
	
	% network interfaces
	\node[anchor=north] at (router-services.north east -| router-description.east) (router-interfaces) {
		\begin{tcbitemize}[interfaces raster]
			\tcbitem[remember as=router-modem] modem\hfill\faIcon{grip-lines}
			\tcbitem[remember as=router-eth0] eth0..3\hfill\faIcon{ethernet}
			\tcbitem[remember as=router-wlan0] wlan0\hfill\faIcon{broadcast-tower}
			\tcbitem[remember as=router-wlan1] wlan1\hfill\faIcon{broadcast-tower}
		\end{tcbitemize}
	};
	
	% draw box around all parts defining a host
	\begin{pgfonlayer}{hostboxes}
		\node[
			host box,
			fit=(router-description.center) (router-services.south) (router-services.west) (router-interfaces.south)
		] (router-host-box) {};
	\end{pgfonlayer}
	% --------------- end router ---------------
	

	% --------------- start printer ---------------
	\node[host description, right=18em of router-description] (printer-description) {%
		\begin{tcolorbox}[
			host description box,
			remember as=printer
		]
		\ip{192.168.0.3}{subnet-home}
		
		\tcblower
		%\deviceImage{my-printer-image}
		
		InkPrint\\
		E-7800
		\end{tcolorbox}
	};
	% --------------- end printer ---------------
	

	% --------------- start raspberrypi ---------------
	% host description
	\node[host description, below=of printer-description] (raspberrypi-description) {%
		\begin{tcolorbox}[
			adjusted title=raspberrypi,
			host description box
		]
		\ip{192.168.0.2}{subnet-home}
		\vspace{\baselineskip}
		
		\begin{tabular}{@{}ll@{}}
			\faIcon{compact-disc} & Pi OS buster \\
			\faIcon{memory} & 1GB \\
			\faIcon{hdd} & 32GB \\
		\end{tabular}
		
		\tcblower
		%\deviceImage{my-raspberry-pi-image}
		
		RPi 3B+
		\end{tcolorbox}
	};
	
	% services
	\node[
		below=0em of raspberrypi-description.south east,
		anchor=north east
	] (raspberrypi-services) {
		\begin{tcboxedraster}[services raster]{services box}
			%\tcbox[enhanced, remember as=raspberrypi-pihole]{\service{pihole}{pi-hole}{\http{pi-hole.internal.lan}}}
			\tcbox[enhanced, remember as=raspberrypi-pihole]{\service{}{pi-hole}{\http{pi-hole.internal.lan}}}
			\begin{tcboxedraster}[services raster]{dockerized box}
				%\tcbox{\service{homeasisstant}{Home Assistant}{\http{hass.internal.lan}}}
				\tcbox{\service{}{Home Assistant}{\http{hass.internal.lan}}}
			\end{tcboxedraster}
		\end{tcboxedraster}
	};

	% service tags
	\node[anchor=east, tag=subnet-home] at (raspberrypi-pihole.east) {DNS};
	
	% network interfaces
	\node[anchor=north] at (raspberrypi-services.north west -| raspberrypi-description.west) (raspberrypi-interfaces) {
		\begin{tcbitemize}[interfaces raster]
			\tcbitem[remember as=raspberrypi-eth0] \faIcon{ethernet}\hfill eth0
		\end{tcbitemize}
	};
	
	% draw box around all parts defining a host
	\begin{pgfonlayer}{hostboxes}
	\node[
		host box,
		fit=(raspberrypi-description.center) (raspberrypi-services.south) (raspberrypi-services.east) (raspberrypi-interfaces.south)
	] (raspberrypi-host-box) {};
	\end{pgfonlayer}
	% --------------- end raspberrypi ---------------
	
	% --------------- start guest devices ---------------
	\node[host description, below=4em of router-host-box.south, anchor=north, inner sep=0em] (guest-devices) {
		\begin{tcboxedraster}[devices raster]{%
			devices box,
			title={Guest Devices\hspace{1em}\ip{192.168.1.0/24}{subnet-guest}},
			remember as=guest-devices-raster}
			%\tcbox{\deviceImageInline{my-guest-device-image} My Guest Device}
			\tcbox{Video Game Console}
			\tcbox{E-Book Reader}
			\tcbox{\textit{... more ...}}
		\end{tcboxedraster}
	};
	% --------------- end guest devices ---------------
	
	
	% --------------- start laptop ---------------
	% host description
	\node[host description, below=of raspberrypi-host-box.south] (laptop-description) {%
		\begin{tcolorbox}[
			adjusted title=laptop,
			host description box
		]
		\ip{192.168.0.4}{subnet-home}
		\vspace{\baselineskip}
		
		\begin{tabular}{@{}ll@{}}
			\faIcon{compact-disc} & Mint 20.3 \\
			\faIcon{memory} & 32GB \\
			\faIcon{hdd} & 1TB \\
		\end{tabular}
		
		\tcblower
		%\deviceImage{my-laptop-image}
		
		Laptop Z680
		\end{tcolorbox}
	};
	
	% services
	\node[
		below=0em of laptop-description.south east,
		anchor=north east
	] (laptop-services) {
		\begin{tcboxedraster}[services raster]{services box}
			%\tcbox{\service{wireguard}{wireguard}{}}
			\tcbox{\service{}{wireguard}{}}
		\end{tcboxedraster}
	};
	
	% network interfaces
	\node[anchor=north] at (laptop-services.north west -| laptop-description.west) (laptop-interfaces) {
		\begin{tcbitemize}[interfaces raster]
			\tcbitem[remember as=laptop-enp0s25] \faIcon{ethernet}\hfill enp0s25
			\tcbitem[remember as=laptop-wlp3s0] \faIcon{broadcast-tower}\hfill wlp3s0
			\tcbitem[remember as=laptop-wg0] \faIcon{dragon}\hfill wg0
		\end{tcbitemize}
	};
	
	% draw box around all parts defining a host
	\begin{pgfonlayer}{hostboxes}
		\node[
			host box,
			fit=(laptop-description.center) (laptop-services.south) (laptop-services.east) (laptop-interfaces.south)
		] (laptop-host-box) {};
	\end{pgfonlayer}
	% --------------- end laptop ---------------
		

	% --------------- start vps ---------------
	% host description
	\node[host description, right=65em of router-description.north east, anchor=north] (vps-description) {%
		\begin{tcolorbox}[
			adjusted title=vps,
			host description box
		]
		\ip{11.22.33.44}{subnet-none}
		\vspace{\baselineskip}
		
		\begin{tabular}{@{}ll@{}}
			\faIcon{compact-disc} & Debian 11 \\
			\faIcon{memory} & 4GB \\
			\faIcon{hdd} & 64GB \\
		\end{tabular}
		
		\tcblower
		%\deviceImage{my-hoster-image}
		
		HyperSpeed Flex Z25
		\end{tcolorbox}
	};
	
	% services
	\node[
		below=0em of vps-description.south east,
		anchor=north east
	] (vps-services) {
		\begin{tcboxedraster}[services raster]{services box}
			\begin{tcboxedraster}[services raster]{dockerized box}
				%\tcbox{\service{wireguard}{wireguard}{}}
				\tcbox{\service{}{wireguard}{}}
				%\tcbox{\service{wordpress}{WordPress \planned{}}{}}
				\tcbox{\service{}{WordPress \planned{}}{}}
			\end{tcboxedraster}
		\end{tcboxedraster}
	};
	
	% network interfaces
	\node[anchor=north] at (vps-services.north west -| vps-description.west) (vps-interfaces) {
		\begin{tcbitemize}[interfaces raster]
			\tcbitem[remember as=vps-eth0] \faIcon{ethernet}\hfill{}eth0
			\tcbitem[remember as=vps-wg0] \faIcon{dragon}\hfill{}wg0
		\end{tcbitemize}
	};
	
	% draw box around all parts defining a host
	\begin{pgfonlayer}{hostboxes}
		\node[
			host box,
			fit=(vps-description.center) (vps-services.south) (vps-services.east) (vps-interfaces.south)
		] (vps-host-box) {};
	\end{pgfonlayer}
	% --------------- end vps ---------------
	

	% --------------- start legend ---------------
	\matrix[
		anchor=north west,
		yshift=2em,
		matrix of nodes,
		row sep=-0.1em,
		every node/.style={anchor=west},
		host box,
		font=\small
	] at (router-description.west |- internet) (legend) {
		& {\Large\sffamily\bfseries Legend} \\
		\draw[ethernet] (0,0) -- (0.75,0);	& Ethernet \\
		\draw[wifi] (0,0) -- (0.75,0);	& Wifi \\
		\draw[vpn, decorate] (0,0) -- (0.75,0);	& VPN \\
		\draw[generic] (0,0) --(0.75,0);	& Other \\
	};
	\node[below=1em of legend.south west, anchor=north west, text width=8em, font=\small] {last updated:\\\today};
	% --------------- end legend ---------------


	% --------------- start network regions ---------------
	\begin{pgfonlayer}{regions}
		% Home
		\node[
			region box=blue,
			fit=(router-host-box.west) (router-description.north) (raspberrypi-host-box.east) (laptop-host-box.south)
		] (home-box) {};
		\node[region descriptor=blue, above, anchor=south west, outer sep=1.5em] at (home-box.south west) {Home};
		
		% Remote
		\node[
			region box=red,
			fit=(vps-eth0.west) (vps-description.north west) (vps-host-box.south east)
		] (remote-box) {};
		\node[region descriptor=red, left, anchor=north east] at (remote-box.south east) {Remote};
	\end{pgfonlayer}
	% --------------- end network regions ---------------


	% --------------- start network connections ---------------
	\begin{scope}[generic]
		\draw (internet) -- (below-internet);
		\draw (isp-home) -| (below-internet);
		\draw (vps-eth0.west) -- ++(-2,0) |- (below-internet) -- ++(0,1);
		\draw (router-modem.east) -- ++(1,0) |- (isp-home);
		
		% connections inside the router
		\draw (router-modem.west) ++(0,0.15) -- ++(-0.5,0) |- (router-wlan1.west);
		\draw foreach \intf in {router-eth0, router-wlan0}
		{(router-modem.west) ++(0,-0.15) -- +(-0.25,0) |- (\intf)};
	\end{scope}

	\begin{scope}[ethernet]
		\draw[subnet-home] (raspberrypi-eth0.west) -- ++(-1,0) |- (router-eth0.east);
		\draw[subnet-home] (printer.west) -- ++(-1.5,0) |- (router-eth0.east);
	\end{scope}

	\begin{scope}[wifi]
		% wifi-guest
		\draw[subnet-guest] (router-wlan1.east) -- ++(1,0) |- (guest-devices-raster.east) node[pos=0.25, above, rotate=90] {\faIcon{wifi} \parbox{7em}{\textbf{wifi-guest, 2.4GHz}}};
		
		% wifi-home
		\draw[subnet-home] (laptop-wlp3s0.west) -- ++(-2,0) |- (router-wlan0.east) node[pos=0.25, below, rotate=90] {\faIcon{wifi} \parbox{7em}{\textbf{wifi-home, 5GHz}}};
	\end{scope}

	\begin{scope}[vpn, subnet-vpn]
		% using large radii on the waves decoration introduces ugly whitespace, so we apply some offset to compensate
		\draw[decorate] (laptop-wg0.west) ++(.3,0) -- ++(-.8,0);		
		\draw[decorate] (vps-wg0.west) ++(.3,0) -- ++(-.8,0);
	\end{scope}
	% --------------- end network connections ---------------
\end{tikzpicture}
	
\end{document}